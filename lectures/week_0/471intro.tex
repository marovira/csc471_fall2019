\documentclass[11pt]{beamer}
\usetheme{Warsaw}
\usepackage[utf8]{inputenc}

\author{Mauricio Andres Rovira Galvez}
\title{CSc 471 - Fall 2019}
\subtitle{Course Introduction}

\setbeamercovered{transparent} 
\setbeamertemplate{navigation symbols}{} 

\institute{University of Victoria \\ Department of Computer Science}  
\date{\today} 
\subject{CSc 471} 

\begin{document}
  \begin{frame}
    \maketitle
  \end{frame}

  \begin{frame}{Overview}
    CSc 471 is an advanced graphics course focusing in computer rendering.
    Specifically, we shall be focusing on real-time rendering using C++ and
    OpenGL. Topics include:
    \begin{itemize}
      \item Rendering pipeline and the GPU,
      \item Shading,
      \item Texturing,
      \item Light and Colour,
      \item Physically-based shading,
      \item Local and Global Illumination,
      \item Rendering optimizations.
    \end{itemize}
  \end{frame}

  \begin{frame}{Computer Science Advising Information}
    \begin{block}{Undergraduate Advisor}
      \begin{itemize}
        \item Irene Statham (\texttt{cscadvisor@uvic.ca})
        \item Office: ECS 512
        \item Undergraduate Advising Hours:
          \begin{itemize}
            \item MWF 10:00-12:00
            \item TR: 13:30-15:30
          \end{itemize}
      \end{itemize}
    \end{block}
    \begin{block}{Administrative Announcements}
      \begin{itemize}
        \item {\tiny If you are taking this course for the third (or
          greater) time, you must request, in \textbf{writing}, permission from
        the Chair of the Department and the Dean of the Faculty.}
        \item {\tiny If you have not met \textbf{all} the prerequisites
            for this course, you must receive department permission to stay in
            this class.  If you do not receive permission, you will be
            \textbf{automatically dropped from the course} and a prerequisite
            drop will be recorded on your transcript}
        \item {\tiny In both of the above cases, you should visit the
          undergraduate advisor for more information.}
      \end{itemize}
    \end{block}
  \end{frame}

  \begin{frame}{Instructor Information}
    \begin{block}{Lectures}
      \begin{itemize}
        \item Mauricio A. Rovira Galvez (\texttt{marovira@uvic.ca})
        \item Lectures: MTR 17:30-18:20 in ECS 108
        \item Office: ECS 621
      \end{itemize}
    \end{block}

    \begin{block}{Labs}
      \begin{itemize}
        \item Mauricio A. Rovira Galvez
        \item Thursdays 18:30-19:20
      \end{itemize}
    \end{block}

    \begin{block}{Mauricio's Office Hours}
      \begin{itemize}
        \item MT 18:30-19:30
      \end{itemize}
    \end{block}
  \end{frame}

  \begin{frame}{conneX Information}
    This course will use the Department of Computer Science's conneX course
    management system for grades and assignments.
    \begin{center}
      \texttt{https://connex.csc.uvic.ca/}
    \end{center}
    Lecture material, code samples, etc will be provided in a separate Github
    page. \textbf{Lecture notes are not comprehensive, and may not include all
    material covered in class. You are responsible for all material covered in
    lectures and labs, including material which is not posted afterwards}.
    \begin{center}
      \texttt{https://github.com/marovira/csc471\_fall2019}
    \end{center}
  \end{frame}

  \begin{frame}{Evaluation Scheme (1)}
    \begin{block}{Assignments}
      Programming Assignments (4): \hspace*{\fill}32\%
    \end{block}

    \begin{block}{Exams}
      Exam 1 (October 28, 2019): \hspace*{\fill}15\% \\
      Exam 2 (December 3, 2019): \hspace*{\fill}15\% \\
    \end{block}

    \begin{block}{Project}
      Programming project : \hspace*{\fill}28\%
    \end{block}

    \begin{block}{In-Class Presentations}
      In-Class Presentations: \hspace*{\fill}10\%
    \end{block}
  \end{frame}

  \begin{frame}{Evaluation Scheme (2)}
    \textbf{Exams:}\\
    Exams 1 and 2 will be 50 minutes long and held during the regular lecture
    time. Both exams will be open-book and open-notes (but no electronic
    devices will be permitted).

    \textbf{Assignments:}\\
    Assignments will be primarily programming based. All assignments will employ
    a graphics development framework.

    \textbf{Programming Project:}\\
    A coding project implementing an advanced topic in rendering. Topics will be
    provided later in the course.

    \textbf{In-Class Presentations:}\\
    A short, 10 minute presentation on an assigned topic in rendering. Topics
    will be provided later in the course.
  \end{frame}

  \begin{frame}{Evaluation Scheme (3)}
    \textbf{Missed Work:} \\
    Exceptions will be made for missed work (including late assignments)
    \textbf{only} in cases where an academic concession (with documentation)
    applies. Links to the relevant university policies are available from the
    posted official course outline. \medskip

    \textbf{Academic Integrity:} \\
    Plagiarism detection software will be used on assignment submissions where
    appropriate. Academic integrity violations will be reported to the
    department's academic integrity committee with recommendations for
    appropriate penalties. Links to the relevant university policies are
    available from the posted official course outline. Note that the
    university's guidelines clearly state that handing an assignment which is
    mostly or entirely plagiarized should result in a grade of F being given for
    the course.
  \end{frame}

  \begin{frame}{Evaluation Scheme (4)}
    \textbf{Acceptable Collaboration:} \\
    Computer Science and Mathematics are inherently collaborative disciplines,
    even if the stereotypes might say otherwise. You are encouraged to discuss
    all aspects of this course, including assignment questions, with your peers.

    However, your actual assignment submissions must be your own work, and
    should be created independently (in your own words). Handing in the work of
    another student and claiming it as your ow in plagiarism. Sharing your
    submission with another student (or the internet), even if it is not
    directly copied by anyone else, is also plagiarism. \medskip

    \textbf{Rule of thumb:} Talk to your peers about assignments and collaborate
    on conceptual solutions, but \textbf{do not} look at each other's code
    (either over the shoulder or by sharing it electronically).
  \end{frame}

\end{document}
